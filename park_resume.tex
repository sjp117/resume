%%%%%%%%%%%%%%%%%%%%%%%%%%%%%%%%%%%%%%%%%
% Medium Length Professional CV
% LaTeX Template
% Version 2.0 (8/5/13)
%
% This template has been downloaded from:
% http://www.LaTeXTemplates.com
%
% Original author:
% Thanks : Rishi Shah 's Contribution
% inspired by his awesome contribution:
% https://www.overleaf.com/articles/rishi-shahs-resume/vgxvkmxktyxn
% Author : Allianzcortex
% contact me : github.com/Allianzcortex
% email : iamwanghz#gmail.com
%
% Current author:
% Author : Sungjoon Park
% github : github.com/sjp117
% email : sjp30117{at}gmail.com
%
% Important note:
% This template requires the resume.cls file to be in the same directory as the
% .tex file. The resume.cls file provides the resume style used for structuring
% the document.
%
%%%%%%%%%%%%%%%%%%%%%%%%%%%%%%%%%%%%%%%%%

%----------------------------------------------------------------------------------------
%	PACKAGES AND OTHER DOCUMENT CONFIGURATIONS
%----------------------------------------------------------------------------------------

\documentclass{resume} % Use the custom resume.cls style

\usepackage[left=0.50in,
			top=0.5in,
			right=0.50in,
			bottom=0.5in]{geometry} % Document margins

\usepackage{fontawesome}
\usepackage{stix2}
\usepackage{hyperref}
\usepackage{hang}
\usepackage{kotex}
\usepackage[style=iso]{datetime2}
%\usepackage{ragged2e}
%\usepackage[bottom]{footmisc} 
\usepackage{enumitem}
\usepackage{fancyhdr}

%----------------------------------------------------------------------------------------
%	HEADER WITH NAME AND CONTACT INFORMATION
%----------------------------------------------------------------------------------------

\name{Sungjoon Park \fontsize{11pt}{12pt}\selectfont 
박성준} % Your name 

\address{
    \faMapMarker{ Pittsburgh, PA}
	\href{https://github.com/sjp117}{\faGithub{ github.com/sjp117}}
%	\href{https://sjp117.gitlab.io}{\faHome{ sjp117.gitlab.io}}
	\href{mailto:sjp30117@gmail.com}{\faEnvelope{ sjp30117@gmail.com}}
	\faPhone{ +1 (979) 422-1264}
}

\begin{document}

\fancyhead[RO,LE]{\today}

%----------------------------------------------------------------------------------------
%	OBJECTIVE SECTION; CONTENT SHOULD CATOR TO EMPLOYER
%----------------------------------------------------------------------------------------

%\begin{rSection}{Objective}
%
%	
%	
%\end{rSection}

%----------------------------------------------------------------------------------------
%	EDUCATION SECTION
%----------------------------------------------------------------------------------------

\begin{rSection}{Education}
    
    {\textbf{Texas A\&M University}, \ College Station, TX \hfill {2020 -- 2022}
    \\ {Master of Science in Psychological Sciences, with Thesis}
    \\ {Supervisor: Dr. Heather Burte}}

    {\textbf{University of Waterloo}, \ Ontario, Canada \hfill 
    {2013 -- 2020}
    \\ {Bachelor of Arts in Psychology, with Thesis, Minor in Philosophy \& Cognitive Science}
    \\ {Thesis Advisor: Dr. Britt Anderson}}

\end{rSection}

%----------------------------------------------------------------------------------------
%	EMPLOYMENT SECTION
%----------------------------------------------------------------------------------------

\begin{rSection}{Employment}
	
	{\textbf{Research Assistant} \hfill {2023 -- Present}
		\\ {Psychology Department}
		\\ {Carnegie Mellon University, \ Pittsburgh, PA}
		\\ {Supervisors: Dr. Laurie Heller \& Dr. Michael Tarr}}
	
\end{rSection}


\begin{rSection}{Research Interest}
	
	{Neural mechanism of spatial cognition and its relationship with non-spatial cognitive functions; Memory; Vision; Neuroimaging; Computational techniques}
	
\end{rSection}



%----------------------------------------------------------------------------------------
%	EXPERIENCE SECTIONS
%----------------------------------------------------------------------------------------

\begin{rSection}{Research Experience}
	
	{\bf \href{https://sites.google.com/andrew.cmu.edu/tarrlab/}{tarrlab}, Carnegie Mellon University}
	\hfill { 2023 - Present}
	\\PI: Dr. Michael Tarr
	\\Roles: Research Assistant
	
	\begin{itemize}[nosep]
%		\setlength\itemsep{-1px}
		
		\item fMRI study participant recruitment, data collection \& processing for Dr. Andrew Luo \& Dr. Maggie Henderson
		\item Setup of computing solutions; Assembling computer, and setting up software
	
	\end{itemize}
	
	{\bf \href{https://www.auditorylab.org/}{Auditory Lab}, Carnegie Mellon University}
	\hfill { 2023 - Present}
	\\PI: Dr. Laurie Heller
	\\Roles: Research Assistant
	
	\begin{itemize}[nosep]
%		\setlength\itemsep{-1px}
		
		\item Data collection, organization, visualization, statistical analysis, and feedback/editing manuscripts for three Misophonia projects.
		\item Naturalistic Multimodal Interaction; Research of sound locallization and the effect of visual and auditory cross-modal interaction (Collaboration with the CMU School of Design; Dr. Daniel Rosenberg Muñoz)
		\item Morphing Everyday Sounds; Development of sound morphing tool that produce psychoacustically informed morphs using machine learning methods (Collaboration with the CMU Computer Science department; Dr. Chris Donahue)
		
	\end{itemize}
    
    {\bf Spatial Thinking \& STEM Learning Lab, 
        Texas A\&M University}
    \hfill { 2020 - 2022}
    \\PI: Dr. Heather Burte
    \\Roles: Graduate Assistant \& Researcher
    
    \begin{itemize}[nosep]
        %\setlength\itemsep{-8px}
        
        \item \emph{Master's Thesis}: Relationship between Perspective Taking 
        with Space and People.
            \begin{itemize}
                \item Used Unity to create virtual environment for study stimuli.
                \item R Scripts, figures, and manuscript available on \href{https://github.com/sjp117/spatialSocialPerspectiveTaking}{GitHub}.
            \end{itemize}
        \item Assisted in formulation of discussion topics and experiments for 
        a Human Cognitive Processes course as part of a Presidential 
        Transformational Teaching Grant project.
        \item Involved in recruitment, management, training and evaluations of research assistants.
        
    \end{itemize}

    {\bf {Human Brain Processes During Complex Locomotor Navigation},
        Texas A\&M University}
    \hfill {2021 - 2022}
    \\PI: Dr. Andrew Nordin \& Dr. Heather Burte
    \\Roles: Graduate Researcher
    
    \begin{itemize}[nosep]
        
        \item The project focused on developing a synchronized mobile EEG and eye tracking system to research brain processes during complex locomotor navigation.
        \item Learned to operate gait (human motion) research hardware \& software (Motek) and analysis pipeline.
        \item Directed and trained 3 undergraduate research assistants to 
        develop experiments using the Python programming language and the Psychopy package.
        \item Involved in the recruitment process of research assistants.
        
    \end{itemize}
    
    {\bf \href{https://brittlab.uwaterloo.ca/}{Britt Lab}, 
        University of Waterloo}
    \hfill {2018 - 2020}
    \\PI: Dr. Britt Anderson
    \\Roles: Research Assistant, Undergraduate Researcher
    
    \begin{itemize}[nosep]
        
        \item \emph{Undergraduate Thesis}: 
        Mental Model Updating and Pupil Response.
            \begin{itemize}
                \item Designed and conducted research on pupillary response to 
                belief updating.
                \item Programmed computerized eye tracking task using Python and the psychopy library.
                \item Self-driven to learn how to program (R \& Python) and use the Linux computing environment.
                \item Took initiative to self-learn and use niche eye-tracking and monitor hardware.
                \item R \& Python codes, and manuscript available on \href{https://github.com/sjp117/Undergrad_Projects/tree/master/mentalModelUpdatingPupil}{GitHub}.
            \end{itemize}
        \item Assisted in two projects that contributed to Master's theses.
        
    \end{itemize}
    
\end{rSection}


%----------------------------------------------------------------------------------------
%	PUBLICATION SECTION
%----------------------------------------------------------------------------------------

\begin{rSection}{Publication}
	\setlength{\hangingindent}{1.27cm}
	
	\begin{hangingpar}
		
		\textbf{Park. S.}, Ferguson. A. J., Mun\~oz. D. R., Heller. L., (\textit{In-preparation}). \href{}{Lateral position discrimination using wavefield synthesis in an open environment to test the effect of observer-controlled motion on the ventriloquist effect.}
		\textit{Auditory Perception \& Cognition.}
		
	\end{hangingpar}
	
	\begin{hangingpar}
		
		Oszczapinska. U., \textbf{Park. S.}, Qiu. Y., Nance. B., Julien. M., Heller. L., (2025). \href{https://doi.org/10.1111/psyp.70014}{The impact of disgusting sounds on pupil diameter of misophonic and non-misophonic listeners.}
		\textit{Psychophysiology.}
		
		
	\end{hangingpar}
	
	\begin{hangingpar}
		
		\textbf{Park. S.}, Watanabe. B., Burte. H., (2022). 
		\href{https://escholarship.org/uc/item/6wq5x6nn}{Perspective taking and reference frames for spatial and social cognition.} 
		\em{Paper submitted to the CogSci 2022 Annual Conference.}
		
	\end{hangingpar}
	
\end{rSection}

\begin{rSection}{Presentation}
	\setlength{\hangingindent}{1.27cm}
	
	\begin{hangingpar}
		
		Ferguson. A. J., \textbf{Park. S.}, Mun\~oz. D. R., Heller. L., (2025, Accepted). \href{}{Lateral position discrimination using wavefield synthesis in an open environment to test the effect of observer-controlled motion on the ventriloquist effect.}
		\em{Poster presented at the
			\href{https://cognitivesciencesociety.org/cogsci-2022/}{joint 188th Meeting of the Acoustical Society of America and 25th International Congress on Acoustics} in New Orleans, Louisiana.}
		
	\end{hangingpar}
	
	\begin{hangingpar}
		
		Henderson M. H., Luo. A. F., \textbf{Park. S.}, Tarr. M. J., Wehbe. L. (2025, Accepted). 
		Generative modeling tools for characterizing human higher visual cortex
		\em{Poster presented at the
			\href{https://www.cogneurosociety.org/poster/?id=6325}{Cognitive Neuroscience Society
				 2025} Annual Meeting.}
		
	\end{hangingpar}
	
	\begin{hangingpar}
		
		Heller. L., Ferguson. A. J., \textbf{Park. S.}, Rosenberg. D. (2024). 
		Naturalistic multimodal spatial interactions.
		\em{Talk presented at the
			\href{https://apcsociety.org/APCAM%202024%20Program.pdf#page=21.10}{23rd Annual Auditory Perception, Cognition, \& Action Meeting.}}
		
	\end{hangingpar}
	
	\begin{hangingpar}
		
		\textbf{Park. S.}, Watanabe. B., Burte. H., (2022). 
		Is Mentalizing Related to Spatial Perspective-Taking? 
		\em{Poster presented at the
			\href{https://www.psychonomic.org/page/2022annualmeeting}{Psychonomic
				Society 2022} Annual Meeting.}
		
	\end{hangingpar}
	
	\begin{hangingpar}
		
		\textbf{Park. S.}, Watanabe. B., Burte. H., (2022). 
		Perspective taking and reference frames for spatial and social 
		cognition. 
		\em{Poster presented to the
			\href{https://cognitivesciencesociety.org/cogsci-2022/}{CogSci 
				2022} Annual Conference.}
		
	\end{hangingpar}
	
	\begin{hangingpar}
		
		Nutalapati. N., Raina. Y., Watanabe. B., \textbf{Park. S.}, \& Burte. 
		H., (2022). 
		How well do you know your campus? A pilot study examining the 
		relationship between anxiety and spatial ability. 
		\em{Poster presented at the Texas A\&M University
			\href{https://srw.tamu.edu/}{Student Research Week} 2022.}
		
	\end{hangingpar}
	
	\begin{hangingpar}
		
		\textbf{Park. S.}, Watanabe. B., Burte. H., (2021). 
		Individual Differences in Perspective Taking for Spatial And Social Cognition. 
		\em{Poster presented at the
			\href{https://www.psychonomic.org/page/2021annualmeeting}{Psychonomic Society 2021} Annual Meeting.}
		
	\end{hangingpar}
	
	\begin{hangingpar}
		
		\textbf{Park. S.}, Watanabe. B., Burte. H., (2021). 
		\href{https://sc2020.lu.lv/wp-content/uploads/2021/08/spatialCog21Poster_SungjoonParkV2.pdf}{Reference Frames for Spatial and Social Thinking: 
			Individual Differences in Strategy Use.} 
		\em{Poster presented at the 
			\href{http://sc2020.lu.lv/}{SPATIAL COGNITION 2020/1} Conference.}
		
	\end{hangingpar}
	
	\begin{hangingpar}
		
		\textbf{Park. S.}, Watanabe. B., Burte. H., (2021). 
		Being good at taking people's spatial perspective might not necessarily 
		mean you are good at “taking their perspective”. 
		\em{Poster presented at the Texas A\&M University
			\href{https://sites.google.com/tamu.edu/2021-2nd-year-poster/home}{Psychological
				and Brain Sciences: 2nd Year (Ph. D) Poster Session} 2021.}
		
	\end{hangingpar}
	
	\begin{hangingpar}
		
		Deshpande. T., \textbf{Park. S.}, Burte. H., (2021). 
		Pointing North Online: Using photographs of known environments to 
		evaluate north pointing accuracy. 
		\em{Poster presented at the CogSci 2021 Annual Conference.}
		
	\end{hangingpar}
	
	\begin{hangingpar}
		
		\textbf{Park. S.}, Anderson. B., (2020). \href{https://github.com/sjp117/Undergrad_Projects/blob/68a2d0fac79b4e8f553d6971f3444f6b2a2f3dfc/mentalModelUpdatingPupil/supplament/psynom20Poster.odp}
		{Mental Model Updating and Pupil Response}. 
		\em{Poster presented at the Virtual Psychonomics 2020 Annual Meeting.}
		
	\end{hangingpar}
	
\end{rSection}

\begin{rSection}{Learning Experience}
	
	\begin{tabular}{ @{} >{}l @{\hspace{3ex}} l }
		Summer 2021 \ & {Computational Neuroscience, 
			\href{https://academy.neuromatch.io/home}{Neuromatch Academy}} \\
		
		Fall 2021 \ & {Basic Training Course on Gait Analysis and Research with the \href{https://www.motekmedical.com/solution/m-gait/}{M-Gait}, Motek}
		
	\end{tabular}
	
\end{rSection}


\begin{rSection}{Teaching Experience}
    
    {\bf {Texas A\&M University}}
    \hfill {2021 - 2022}
    \\Graduate Teaching Assistant, 4 Semesters
    
    \begin{itemize}[nosep]
        
        \item Assisted and instructed labs for undergraduate
        psychology statistics, research methods, and scientific
        writing courses.
        
    \end{itemize}

    Lab Trainer, 3 Semesters
    
    \begin{itemize}[nosep]
        
        \item Trained, up to 9, undergraduate research assistants per 
        semester in a research lab on how to use R and RStudio.
        \item Students learned to produce descriptive statistics, visualizations, and conduct correlation, t-tests, and ANOVA tests.
        
    \end{itemize}
    
\end{rSection}

%\break
%


%----------------------------------------------------------------------------------------
%	SKILLS SECTION
%----------------------------------------------------------------------------------------

\begin{rSection}{Skills}
    
    \begin{tabular}{ @{} >{\bfseries}l @{\hspace{3ex}} l }
        
        Programming: \ & R, Python, C++ \\%, Bash \\
        Software \& Tools: 
        & Linux, CLI, git, Remote tools (i.e., ssh, rsync, etc...), SPSS, Microsoft Office, Qualtrics, \LaTeX \\
        fMRI: \ & FSL, fmriprep, pycortex, freesurfer\\
        Statistics: \ & Regression (linear, generalized, multiple, hierarchical), Mixed Effects Model, \\
        \ & Correlation, t-test, ANOVA, Descriptive Statistics, contrast analysis \\
        Hardware: & {\textbf{Eye Tracking: }}SR Research  EyeLink 1000 Plus, CRS ltd. LiveTrack, SmartEye Aurora \\
        & {\textbf{Gait Tracking: }}Motek M-Gait \\
        Languages: \ &  English, Korean\\
        
    \end{tabular}
    
\end{rSection}

%----------------------------------------------------------------------------------------
%	WORK ELIGIBILITY SECTION; FOR F-1 VISA
%----------------------------------------------------------------------------------------

%\begin{rSection}{Work Authorization}
%    
%    % duration should vary with opt / STEM extention eligibility
%    Eligible to work in the U.S. without sponsorship for 36 months with 
%    Optional Practical Training (OPT) \& STEM Extension. Expires: 01/30/2026.
%    
%\end{rSection}

\end{document}
