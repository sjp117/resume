%%%%%%%%%%%%%%%%%%%%%%%%%%%%%%%%%%%%%%%%%
% Medium Length Professional CV
% LaTeX Template
% Version 2.0 (8/5/13)
%
% This template has been downloaded from:
% http://www.LaTeXTemplates.com
%
% Original author:
% Thanks : Rishi Shah 's Contribution
% inspired by his awesome contribution:
% https://www.overleaf.com/articles/rishi-shahs-resume/vgxvkmxktyxn
% Author : Allianzcortex
% contact me : github.com/Allianzcortex
% email : iamwanghz#gmail.com
%
% Current author:
% Author : Sungjoon Park
% github : github.com/sjp117
% email : sjp30117{at}gmail.com
%
% Important note:
% This template requires the resume.cls file to be in the same directory as the
% .tex file. The resume.cls file provides the resume style used for structuring
% the document.
%
%%%%%%%%%%%%%%%%%%%%%%%%%%%%%%%%%%%%%%%%%

%----------------------------------------------------------------------------------------
%	PACKAGES AND OTHER DOCUMENT CONFIGURATIONS
%----------------------------------------------------------------------------------------

\documentclass{resume} % Use the custom resume.cls style

\usepackage[left=0.50in,
top=0.5in,
right=0.50in,
bottom=0.5in]{geometry} % Document margins

\usepackage{fontawesome}
\usepackage{stix2}
\usepackage{hyperref}
\usepackage{hang}
\usepackage{kotex}
\usepackage[style=iso]{datetime2}
\usepackage{ragged2e}
%\usepackage[bottom]{footmisc} 
\usepackage{enumitem}
\usepackage{fancyhdr}

%----------------------------------------------------------------------------------------
%	HEADER WITH NAME AND CONTACT INFORMATION
%----------------------------------------------------------------------------------------

\name{Sungjoon Park \fontsize{11pt}{12pt}\selectfont 
    박성준} % Your name 

\address{
    %\faMapMarker{ Pittsburgh, PA}
    \href{https://github.com/sjp117}{\faGithub{ github.com/sjp117}}
    \href{mailto:sjp30117@gmail.com}{\faEnvelope{ sjp30117@gmail.com}}
    %\faPhone{ +1 (979) 422-1264}
}

\begin{document}
    
    \fancyhead[RO,LE]{\today}
    
    %----------------------------------------------------------------------------------------
    %	PROFESSIONAL SUMMARY SECTION
    %----------------------------------------------------------------------------------------
    
    \begin{rSection}{Professional Summary}
        {Cognitive neuroscience researcher with 5+ years of experience. 
            Specializing in behavioral, fMRI, 
            eye-tracking, and computational methods. Proficient in experimental 
            design, cross-modal research, and statistical 
            modeling of cognitive and behavioral processes. Strong record of 
            collaborative 
            research across 8 research projects resulting in 4 peer-reviewed 
            publications and 11 conference presentations. Committed to 
            advancing 
            understanding of neural mechanisms underlying spatial and social 
            cognition through innovative methodology and interdisciplinary 
            collaboration.}
    \end{rSection}
    
    %----------------------------------------------------------------------------------------
    %	PROFESSIONAL EXPERIENCE SECTION
    %----------------------------------------------------------------------------------------
    
    \begin{rSection}{Professional Experience}
        
        {\textbf{Staff Research Assistant} \hfill {2023 -- Present}
            \\ {Carnegie Mellon University, Pittsburgh, PA}}
        
        \begin{itemize}[nosep, leftmargin=*, widest=0]
            \item Involved in 6 projects between 2 labs. Contributed to 4 
            peer-reviewed publications and 3 
            presentations.
            \item Processed and analyzed fMRI data using FSL, 
            fmriprep, and adapted custom codes, for mapping neural selectivity 
            in human visual cortex.
            \item Developed data visualization, wrangling, and statistical 
            analysis pipeline using R and 
            Python to effectively interpret data and communicate statistical 
            inferences.
            \item Coordinated data collection across 6 behavioral, cognitive, 
            and neuroimaging studies. Both online and in-person.
            \item Implemented R code for data wrangling and statistical 
            analyses of behavioral and cognitive data.
            \item Collaborated with computer scientists to develop machine 
            learning algorithms for sound morphing tools. Used Python to 
            generate sounds and tested human evaluations of sound morphs.
            \item Collaborated with designers to study human spatial sound 
            localization.
        \end{itemize}
        
        {\textbf{Graduate Research Assistant} \hfill {2020 -- 2022}
            \\ {Texas A\&M University, College Station, TX}}
        
        \begin{itemize}[nosep, leftmargin=*, widest=0]
            \item Designed and executed 
            master's thesis examining cognitive 
            mechanisms shared between spatial and social perspective-taking. 
            Designed and conducted 2 experiments: one online, and 
            one 
            in-person experiment.
            \item Created Unity-based virtual environments for experimental 
            stimuli.
            \item Published thesis findings establishing connections 
            between spatial and social cognitive processes, with portions 
            published on a peer-reviewed conference proceedings.
            \item Trained up to 9 undergraduate research assistants per 
            semester on how to program using R, RStudio, and how to conduct 
            basic data wrangling and statistical analyses.
            \item Collaborated with a kinesiology team to research brain 
            processes during complex locomotor navigation.
            \item Supervised a team of 3 undergraduate research 
            assistants, teaching how to program an experiment using Python, 
            resulting in a pilot study on spatial memory and resulting in a 
            poster presentation.
        \end{itemize}
        
        {\textbf{Undergraduate Researcher} \hfill {2018 -- 2020}
            \\ {University of Waterloo, Ontario, Canada}}
        
        \begin{itemize}[nosep, leftmargin=*, widest=0]
            \item Designed and programmed eye-tracking experiment using 
            Python that measured pupillary response to how we change 
            our minds with new evidence.
            \item Built custom data analysis pipeline that processed 
            of eye-tracking data.
            \item Self-motivated learning of Linux, R, and Python, developing 
            computational skills used in all subsequent research projects.
            \item Assisted in data collection that contributed to a Master's 
            thesis.
        \end{itemize}
        
    \end{rSection}
    \pagebreak
    %----------------------------------------------------------------------------------------
    %	SKILLS SECTION
    %----------------------------------------------------------------------------------------
    
    \begin{rSection}{Skills}
        
        \begin{tabular}{ @{} >{\bfseries}l @{\hspace{3ex}} 
                >{\RaggedRight}p{0.775\textwidth} }
            Data Analysis: \ & Data wrangling, visualization, statistical 
            modeling (mixed effects, generalized), Correlation, t-test, ANOVA, 
            descriptive statistics, contrast analysis, repeated-measures \\
            Software: \ & R, Python, SPSS, Microsoft Office, C++ (hobby) \\
            Computing: \ & Linux environment, version control (git), remote 
            computing, bash scripting \\
            Research Methods: \ & fMRI, eye-tracking experiment design, 
            cross-modal research methodology \\
            Neuroimaging: \ & FSL, fmriprep, pycortex, freesurfer \\
            Hardware: \ & Eye-tracking systems (EyeLink, LiveTrack, SmartEye), 
            Motion capture (Motek M-Gait), Personal Computers \\
            Soft Skills: \ & Interdisciplinary collaboration, mentoring, 
            technical communication, project management \\
            Languages: \ & English (Native), Korean (Native)
        \end{tabular}
        
    \end{rSection}
    
    %----------------------------------------------------------------------------------------
    %	PUBLICATION SECTION
    %----------------------------------------------------------------------------------------
    
    \begin{rSection}{Selected Publications \& Presentations}
        \setlength{\hangingindent}{1.27cm}
        
        %\begin{hangingpar}
        %   \textbf{Park. S.}, Ferguson. A. J., Rosenberg. D. M., Heller. L., 
        %    (\textit{In-preparation}). 
        %    \href{}{Lateral position discrimination using wavefield synthesis 
            %    in an open environment to test the effect of 
            %observer-controlled 
            %    motion on the ventriloquist effect.}
        %    \textit{Auditory Perception \& Cognition.}
        %\end{hangingpar}
        
        \begin{hangingpar}
            Oszczapinska. U., \textbf{Park. S.}, Qiu. Y., Nance. B., Julien. 
            M., Heller. L., (2025). 
            \href{https://doi.org/10.1111/psyp.70014}{The impact of disgusting 
                sounds on pupil diameter of misophonic and non-misophonic 
                listeners.}
            \textit{Psychophysiology.}
        \end{hangingpar}
        
        \begin{hangingpar}
            Henderson M. H., Luo. A. F., \textbf{Park. S.}, Tarr. M. J., Wehbe. 
            L. (2025). \href{https://www.cogneurosociety.org/poster/?id=6325}{
                Generative modeling tools for characterizing human higher 
                visual 
                cortex.}
            \em{Poster presented at the Cognitive Neuroscience Society 2025 
                Annual Meeting.}
        \end{hangingpar}
        
        \begin{hangingpar}
            \textbf{Park. S.}, Watanabe. B., Burte. H., (2022). 
            \href{https://escholarship.org/uc/item/6wq5x6nn}{Perspective taking 
                and reference frames for spatial and social cognition.} 
            \em{Paper submitted to the CogSci 2022 Annual Conference.}
        \end{hangingpar}
        
    \end{rSection}
    
    %----------------------------------------------------------------------------------------
    %	EDUCATION SECTION
    %----------------------------------------------------------------------------------------
    
    \begin{rSection}{Education}
        
        {\textbf{Texas A\&M University}, \ College Station, TX \hfill {2020 -- 
                2022}
            \\ {Master of Science in Psychological Sciences} 
            \\ Thesis: \textit{Relationship between Perspective Taking with 
                Space and People}
            %\\ Relevant Coursework: Neuroimaging Data Analysis, Eye 
            %Tracking and Visual Perception, Applied Structural Equation 
            %Modeling
        }
        
        {\textbf{University of Waterloo}, \ Ontario, Canada \hfill 
            {2013 -- 2020}
            \\ {Bachelor of Arts in Psychology, with Thesis, Minor in 
                Philosophy \& Cognitive Science}
            \\ Thesis: \textit{Mental Model Updating and Pupil Response}
            %\\ Relevant Coursework: Computational Neuroscience
            %Methods, Advanced Data Analysis, Advanced Statistics
        }
        
    \end{rSection}
    
\end{document}