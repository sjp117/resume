%%%%%%%%%%%%%%%%%%%%%%%%%%%%%%%%%%%%%%%%%
% Medium Length Professional CV
% LaTeX Template
% Version 2.0 (8/5/13)
%
% This template has been downloaded from:
% http://www.LaTeXTemplates.com
%
% Original author:
% Thanks : Rishi Shah 's Contribution
% inspired by his awesome contribution:
% https://www.overleaf.com/articles/rishi-shahs-resume/vgxvkmxktyxn
% Author : Allianzcortex
% contact me : github.com/Allianzcortex
% email : iamwanghz#gmail.com
%
% Current author:
% Author : Sungjoon Park
% github : github.com/sjp117
% email : sjp30117{at}gmail.com
%
% Important note:
% This template requires the resume.cls file to be in the same directory as the
% .tex file. The resume.cls file provides the resume style used for structuring
% the document.
%
%%%%%%%%%%%%%%%%%%%%%%%%%%%%%%%%%%%%%%%%%

%----------------------------------------------------------------------------------------
%	PACKAGES AND OTHER DOCUMENT CONFIGURATIONS
%----------------------------------------------------------------------------------------

\documentclass{resume} % Use the custom resume.cls style

\usepackage[left=0.50in,
			top=0.5in,
			right=0.50in,
			bottom=0.5in]{geometry} % Document margins

\usepackage{fontawesome}
\usepackage{stix2}
\usepackage{hyperref}
\usepackage{hang}
\usepackage{kotex}
\usepackage[style=iso]{datetime2}
%\usepackage{ragged2e}
\usepackage[bottom]{footmisc} 
\usepackage{enumitem}

%----------------------------------------------------------------------------------------
%	HEADER WITH NAME AND CONTACT INFORMATION
%----------------------------------------------------------------------------------------

\name{Sungjoon Park \fontsize{11pt}{12pt}\selectfont 
박성준\rlap{\hspace*{1.7in}\normalfont\today}} % Your name 

\address{
    \faMapMarker{ College Station, TX}
	\href{https://github.com/sjp117}{\faGithub{ github.com/sjp117}}
%	\href{https://sjp117.gitlab.io}{\faHome{ sjp117.gitlab.io}}
	\href{mailto:sungjoon.park@tamu.edu}{\faEnvelope{ sungjoon.park@tamu.edu}}
	\faPhone{ +1 (979) 422-1264}
}

\begin{document}
    
%----------------------------------------------------------------------------------------
%	OBJECTIVE SECTION; CONTENT SHOULD CATOR TO EMPLOYER
%----------------------------------------------------------------------------------------

\begin{rSection}{Objective}

	
	
\end{rSection}

%----------------------------------------------------------------------------------------
%	EDUCATION SECTION
%----------------------------------------------------------------------------------------

\begin{rSection}{Education}
    
    {\textbf{Texas A\&M University}, \ College Station, TX \hfill {December 
    2022}
    \\ \textit {Master of Science in Experimental Psychology, with Thesis}}

    {\textbf{University of Waterloo}, \ Ontario, Canada \hfill 
    {June 2020}
    \\ \textit {Bachelor of Arts in Psychology, Minor in Philosophy, Cognitive 
    Science}}

\end{rSection}

%----------------------------------------------------------------------------------------
%	EXPERIENCE SECTIONS
%----------------------------------------------------------------------------------------

\begin{rSection}{Research Experience}
    
    {\bf \href{https://burtelab.sites.tamu.edu/}{Spatial Thinking \& STEM 
    Learning Lab}, 
        Texas A\&M University}
    \hfill { 2020 - Present}
    \\PI: Dr. Heather Burte
    \\Roles: Graduate Assistant \& Researcher
    
    \begin{itemize}[nosep]
        %\setlength\itemsep{-8px}
        
        \item \emph{Master's Thesis}: Relationship between Perspective Taking 
        with Space and People.
            \begin{itemize}
                \item Designed and conducted research to examine the use of 
                perspective taking 
                ability in spatial and social (mentalizing) context.
            \end{itemize}
        \item Research into evaluating North pointing ability using photographs 
        of the Texas A\&M University Campus.
        \item Assisted in formulation of discussion topics and experiments for 
        a Human Cognitive Processes course as part of a Presidential 
        Transformational Teaching Grant project.
        
    \end{itemize}
    
    {\bf {Human Brain Processes During Complex Locomotor Navigation},
        Texas A\&M University}
    \hfill {2021 - 2022}
    \\PI: Dr. Andrew Nordin \& Dr. Heather Burte
    \\Roles: Graduate Researcher
    
    \begin{itemize}[nosep]
        
        \item Research into how the brain is able to process information for 
        navigating complex terrain.
        \item Develop new methods for wirelessly measuring human brain and body 
        dynamics during gait.
        \item Directed and trained 3 undergraduate research assistants to 
        develop computerized experiments.
        
    \end{itemize}
    
    {\bf \href{https://brittlab.uwaterloo.ca/}{Britt Anderson Group}, 
        University of Waterloo}
    \hfill {2018 - 2020}
    \\PI: Dr. Britt Anderson
    \\Roles: Research Assistant, Undergraduate Researcher
    
    \begin{itemize}[nosep]
        
        \item \emph{Undergraduate Thesis}: 
        \href{https://github.com/sjp117/Undergrad_Projects/tree/master/mentalModelUpdatingPupil}{Mental
         Model Updating and Pupil Response.}
            \begin{itemize}
                \item Designed and conducted research on pupillary response to 
                belief change.
            \end{itemize}
        \item Assisted in two experiments that contributed to Master's theses.
        
    \end{itemize}
    
\end{rSection}


\begin{rSection}{Teaching Experience}
    
    {\bf {Texas A\&M University}}
    \hfill {2021 - 2022}
    \\Graduate Teaching Assistant, 4 Semesters
    
    \begin{itemize}[nosep]
        
        \item Assisted and instructed labs for undergraduate psychology 
        statistics and research methods courses.
        
    \end{itemize}

    Lab Trainer, 2.5 Semesters
    
    \begin{itemize}[nosep]
        
        \item Trained, up to 9, undergraduate research assistants per 
        semester in the Spatial Thinking \& STEM Learning Lab.
        \item Introductory R Tutorial.
        
    \end{itemize}
    
\end{rSection}

%\break

\begin{rSection}{Learning Experience}
    
    \begin{tabular}{ @{} >{}l @{\hspace{3ex}} l }
        
        Fall 2021 \ & {Basic Training Course on Gait Analysis and Research with 
        the \href{https://www.motekmedical.com/solution/m-gait/}{M-Gait}, Motek}
        
        \\
        
        Summer 2021 \ & {Computational Neuroscience, 
        \href{https://academy.neuromatch.io/home}{Neuromatch Academy}}
        
    \end{tabular}
    
\end{rSection}

%----------------------------------------------------------------------------------------
%	SKILLS SECTION
%----------------------------------------------------------------------------------------

\begin{rSection}{Technical Skills}
    
    \begin{tabular}{ @{} >{\bfseries}l @{\hspace{3ex}} l }
        
        Programming: \ & R, Python \\
        Software \& Tools: 
        & Qualtrics, Psychopy, FSL, Linux, Command Line, git, \LaTeX\\
        Hardware: & {\textbf{Eye Tracking: }}CRS ltd. LiveTrack, SR Research 
        EyeLink 1000 Plus \\
        & {\textbf{Gait Tracking: }}Motek M-Gait \\
        Languages: \ &  English, Korean\\
        
    \end{tabular}
    
\end{rSection}

%----------------------------------------------------------------------------------------
%	PUBLICATION SECTION
%----------------------------------------------------------------------------------------

\begin{rSection}{Publication}
	\setlength{\hangingindent}{1.27cm}
    
    \begin{hangingpar}
        
        \textbf{Park. S.}, Watanabe. B., Burte. H., (in-preparation). 
        Title: TBD. 
        \em{Paper submitted to the
            \href{https://cognitivesciencesociety.org/cogsci-2023/}{CogSci 
            2023} Annual Conference.}
        
    \end{hangingpar}
	
	\begin{hangingpar}
		
		\textbf{Park. S.}, Watanabe. B., Burte. H., (2022). 
		Perspective taking and reference frames for spatial and social cognition. 
		\em{Paper submitted to the
			\href{https://cognitivesciencesociety.org/cogsci-2022/}{CogSci 2022} Annual Conference.}
		
	\end{hangingpar}
	
\end{rSection}

\begin{rSection}{Presentation}
	\setlength{\hangingindent}{1.27cm}
    
    \begin{hangingpar}
        
        \textbf{Park. S.}, Watanabe. B., Burte. H., (2022). 
        Is Mentalizing Related to Spatial Perspective-Taking? 
        \em{Poster presented at the
            \href{https://www.psychonomic.org/page/2022annualmeeting}{Psychonomic
                Society 2022} Annual Meeting.}
        
    \end{hangingpar}

    \begin{hangingpar}
        
        \textbf{Park. S.}, Watanabe. B., Burte. H., (2022). 
        Perspective taking and reference frames for spatial and social 
        cognition. 
        \em{Poster presented to the
            \href{https://cognitivesciencesociety.org/cogsci-2022/}{CogSci 
            2022} Annual Conference.}
        
    \end{hangingpar}

    \begin{hangingpar}
        
        Nutalapati. N., Raina. Y., Watanabe. B., \textbf{Park. S.}, \& Burte. 
        H., (2022). 
        How well do you know your campus? A pilot study examining the 
        relationship between anxiety and spatial ability. 
        \em{Poster presented at the Texas A\&M University
            \href{https://srw.tamu.edu/}{Student Research Week} 2022.}
        
    \end{hangingpar}
	
	\begin{hangingpar}
		
		\textbf{Park. S.}, Watanabe. B., Burte. H., (2021). 
		Individual Differences in Perspective Taking for Spatial And Social Cognition. 
		\em{Poster presented at the
			\href{https://www.psychonomic.org/page/2021annualmeeting}{Psychonomic Society 2021} Annual Meeting.}
		
	\end{hangingpar}
	
	\begin{hangingpar}
		
		\textbf{Park. S.}, Watanabe. B., Burte. H., (2021). 
		\href{https://sc2020.lu.lv/wp-content/uploads/2021/08/spatialCog21Poster_SungjoonParkV2.pdf}{Reference Frames for Spatial and Social Thinking: 
		Individual Differences in Strategy Use.} 
		\em{Poster presented at the 
			\href{http://sc2020.lu.lv/}{SPATIAL COGNITION 2020/1} Conference.}
		
	\end{hangingpar}

    \begin{hangingpar}
        
        \textbf{Park. S.}, Watanabe. B., Burte. H., (2021). 
        Being good at taking people's spatial perspective might not necessarily 
        mean you are good at “taking their perspective”. 
        \em{Poster presented at the Texas A\&M University
            \href{https://sites.google.com/tamu.edu/2021-2nd-year-poster/home}{Psychological
             and Brain Sciences: 2nd Year (Ph. D) Poster Session} 2021.}
        
    \end{hangingpar}
	
	\begin{hangingpar}
		
		 Deshpande. T., \textbf{Park. S.}, Burte. H., (2021). 
		 Pointing North Online: Using photographs of known environments to 
		 evaluate north pointing accuracy. 
		 \em{Poster presented at the CogSci 2021 Annual Conference.}
		
	\end{hangingpar}

	\begin{hangingpar}
		
		\textbf{Park. S.}, Anderson. B., (2020). \href{https://brittlab.uwaterloo.ca/2021/01/07/sungjoon-presents-psychonomics/}
		{Mental Model Updating and Pupil Response}. 
		\em{Poster presented at the Virtual Psychonomics 2020 Annual Meeting.}
		
	\end{hangingpar}
	
\end{rSection}

%----------------------------------------------------------------------------------------
%	WORK ELIGIBILITY SECTION; FOR F-1 VISA
%----------------------------------------------------------------------------------------

\begin{rSection}{Work Authorization}
    
    % duration should vary with opt / STEM extention eligibility
    Eligible to work in the U.S. without sponsorship for 36 months with 
    Optional Practical Training (OPT).
    
\end{rSection}

\end{document}
